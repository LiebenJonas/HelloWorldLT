%!TEX root = ../dokumentation.tex

\chapter{Einleitung}
In der modernen Automobilindustrie spielen Fahrassistenzsysteme eine zentrale Rolle bei der Verbesserung von Fahrsicherheit und Fahrkomfort. Diese Systeme können nur funktionieren, weil das moderne Automobil über eine Vielzahl verschiedenster Sensoren verfügt, welche die notwendigen Daten schnell und verlässlich ermitteln. \protect\citeM{i}{Winner_Hermann_Hakuli_Stephan_Wolf_Gabriele.2009}{Einleitung}{Einleitung}, \protect\citeM{i}{Tille_Thomas.2016}{Einleitung}{Einleitung}\\
Diese Arbeit beschäftigt sich mit der Implementierung eines Systems zum bereitstellen von Livedaten eines Abstandssensors. Hierzu wird ein Mikrocontroller (\autoref{sec:Mikrocontroller}) des Typs ESP8266 (\autoref{subsec:ESP8266}) verwendet. Dieser fungiert als zentrale Steuereinheit eines angeschlossenen Abstandssensors und übernimmt die Erfassung von Livedaten sowie auch die Bereitsstellung dieser mittels eines Webservers. Zum Zwecke der Nutzerfreundlichkeit sollen über den Webserver nicht nur die aktuellen Daten in einem sinnvollen Format, sondern auch eine Website bereitgestellt werden, welche die Daten Live für den Anwender visualisiert.  

\section{Thema}
TODO

\section{Fragestellung}
TODO

\section{Zielsetzung}
Zielsetzung des Hello-Word Projektes ist die Nachstellung Gewinnung von Sensordaten über einen ESP8266. Zudem soll der ESP8266 sich mit einem lokalen WLAN verbinden und als Webserver fungieren, welchen man zum Erhalt der Daten ansprechen kann.

\section{Aufbau}
Aufgabe des {\arbeit}-Projektes ist es, mithilfe eines ESP8266 ein System zu entwickeln, welches Sensordaten in einem Mercedes-Benz Fahrzeug erfasst und über einen Webserver im Rahmen einer Website als Live-Daten zur verfügung stellt. Dieses Projekt dient später als Grundlage für die in der Uni abzugebende T1000-Dokumentation.

\section{Probleme (gehört hier nicht her)}
\begin{itemize}
	\item Wie schafft man es, dass die Daten aktuell auf einer Website angezeigt werden
	\subitem Übergabe von HTML, JS und CSS, wobei JS-schicht die Daten live abfragt und auf der Seite visualisiert
    \item Treiber-Installation auf Firmenrechner unmöglich -> Eigener Rechner benötigt
    \item Wie verwende ich den Sensor?
\end{itemize}