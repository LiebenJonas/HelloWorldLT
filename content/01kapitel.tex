%!TEX root = ../dokumentation.tex

\chapter{Einleitung}
In modernen Fahrzeugen spielen Sensordaten eine zentrale Rolle. Egal ob bei der Steigerung der Effizienz, bei der Verringerung von Risiken, oder bei der Verbesserung des Fahrerlebnisses: Sensordaten sind ein zentraler Teil des modernen Automobils.  


Schon 1982 wurde im Toyota Corona die Erste auf Sensordaten basierende, aktive Einparkhilfe verbaut \protect\citeM{i}{Online.2022}{22}{28}. 
Beispielsweise wurde im Toyota Corona bereits 1982 die Erste auf Sensordaten basierende, aktive Einparkhilfe verbaut. Diese verwendete Ultraschallsensoren, um die Distanz zum nächsten Objekt nach Hinten zu messen und den Fahrer wie heute fast selbsverständlich akustisch auf den Abstand hinzuweisen.



\section{Thema}
TODO

\section{Fragestellung}
TODO

\section{Zielsetzung}
Zielsetzung des Hello-Word Projektes ist die Nachstellung Gewinnung von Sensordaten über einen ESP8266. Zudem soll der ESP8266 sich mit einem lokalen WLAN verbinden und als Webserver fungieren, welchen man zum Erhalt der Daten ansprechen kann.

\section{Aufbau}
Aufgabe des {\arbeit}-Projektes ist es, mithilfe eines ESP8266 ein System zu entwickeln, welches Sensordaten in einem Mercedes-Benz Fahrzeug erfasst und über einen Webserver im Rahmen einer Website als Live-Daten zur verfügung stellt. Dieses Projekt dient später als Grundlage für die in der Uni abzugebende T1000-Dokumentation.

\section{Probleme (gehört hier nicht her)}
\begin{itemize}
	\item Wie schafft man es, dass die Daten aktuell auf einer Website angezeigt werden
	\subitem Übergabe von HTML, JS und CSS, wobei JS-schicht die Daten live abfragt und auf der Seite visualisiert
    \item Treiber-Installation auf Firmenrechner unmöglich -> Eigener Rechner benötigt
    \item Wie verwende ich den Sensor?
\end{itemize}