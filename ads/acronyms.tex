%!TEX root = ../dokumentation.tex
\addchap{\langabkverz}
%nur verwendete Akronyme werden letztlich im Abkürzungsverzeichnis des Dokuments angezeigt
%Verwendung: 
%		\ac{Abk.}   --> fügt die Abkürzung ein, beim ersten Aufruf wird zusätzlich automatisch die ausgeschriebene Version davor eingefügt bzw. in einer Fußnote (hierfür muss in header.tex \usepackage[printonlyused,footnote]{acronym} stehen) dargestellt
%		\acs{Abk.}   -->  fügt die Abkürzung ein
%		\acf{Abk.}   --> fügt die Abkürzung UND die Erklärung ein
%		\acl{Abk.}   --> fügt nur die Erklärung ein
%		\acp{Abk.}  --> gibt Plural aus (angefügtes 's'); das zusätzliche 'p' funktioniert auch bei obigen Befehlen
%       \aclp{Abk.} --> gibt die Erklärung in Plura (angefügtes 's' aus
%	siehe auch: http://golatex.de/wiki/%5Cacronym
%	



\begin{acronym}[YTMMM]
\setlength{\itemsep}{.5\parsep}

\acro{API}{App Programming Interface}
\acro{DHBW}{Duale Hochschule Baden-Würtemberg}
\acro{JSON}{Javascript Object Notation}
\acro{ISO}{International Standardization Organisation}
\acro{OSI}{Open Systems Interconnect}
\acro{HTML}{Hypertext Markup Language}
\acro{HTTP}{Hypertext Transfer Protocol}
\acro{HTTPS}{Hypertext Transfer Protocol Secure}
\acro{LAN}{Local Area Network}
\acro{WLAN}{Wireless \ac{LAN}}
\acro{ADAS}{Advanced Driver Assistance System}
\acro{LiDAR}{Light Detection and Ranging}

\end{acronym}
